% Created by Jim Finnis
% Date Mon Jun 14 12:18:40 2021


\section{Customising PCOT}
\label{customisation}
After running PCOT for the first time, each user will have a
\texttt{.pcot.ini} file in the their home directory for settings
and some persistent data. This
section of the document describes this data.

\subsection{Locations section}
\begin{itemize}
\item \textbf{pcotfiles}: location of PCOT documents; the directory
of the last PCOT file saved or loaded.
\item \textbf{images}: the directory of the last image loaded.
\item \textbf{mplplots}: the directory of the last MPL plot saved.
\item \textbf{pluginpath}: a colon-separated list of directories
scanned for plugins (see below).
\end{itemize}

\subsection{Plugins}
Each directory inside the plugin path (see above) is scanned at
application startup, on import of the \texttt{pcot} module. All
Python files found are imported. Various mechanisms are provided
to let users create their own nodes, \emph{expr} functions and
menu items.

\subsubsection{Creating new types of node}
Simply adding a new subclass of \texttt{XFormType} preceded by
the \texttt{@xformtype} decorator is sufficient to register
a new node type. See Sec.~\ref{writingnodes} for more details,
or read some of PCOT's source code --- node types can be found
in the \texttt{xforms} directory.

\subsubsection{Creating new \emph{expr} functions}
This is done by adding an expression function hook to the list of hooks. These
hooks are functions which take an \texttt{ExpressionEvaluator} class. Add something
like the following to one of your Python files in your plugin directory:
\begin{lstlisting}
import pcot

def testfunc(args, optargs):
    # just a numeric test function that calculates a+b*2 for two parameters a,b
    # get the two values as numbers
    a = args[0].get(conntypes.NUMBER)
    b = args[1].get(conntypes.NUMBER)
    # calculate the result (yes, I'm doing this the long-winded way)
    result = a+b*2
    # convert the result into a Datum and return
    return Datum(conntypes.NUMBER, result)

def regfuncs(e):
    from pcot.expression.parse import Parameter
    # register a function "test" with a description
    p.registerFunc("test", "func description",
    # one entry for each mandatory argument/parameter
                   [Parameter("a", "description for A", conntypes.NUMBER),
                    Parameter("b","description for B", conntypes.NUMBER)
                   ],
                   [], # no optional arguments
                   testfunc # finally, the fnction
                   )

# register the function by adding it to the expression function hooks - this
# will run regfuncs() when the expression evaluator is created.

pcot.exprFuncHooks.append(regfuncs)
\end{lstlisting}
See~\ref{writingfuncs} for more details.
