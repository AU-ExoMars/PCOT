% Created by Jim Finnis
% Date Wed Dec 22 10:28:57 2021


\clearpage
\section{Installation}
\label{inst:conda}
The following instructions are intended for developers or users who
wish to use PCOT as a library, or to develop their own plugins (typically
XForm node types or \emph{expr} node functions).
Other users receive a single-file executable constructed by PyInstaller
using the instructions below. This does not require installation --- just
running the file will run PCOT. Different executables are required for
each platform.

Firstly, you will need to install Anaconda (\url{https://docs.anaconda.com/anaconda/install/index.html}). You may already have it
installed. Then you'll need to open an command line into which you can enter Anaconda commands --- this is platform dependent; on
Linux it's just a normal shell (and I think this is also true on MacOS), while on Windows it will be a program in the Anaconda package.
The following commands will 
create a new Anaconda environment called \texttt{pcot} and install the required packages into it:
\vspace{1em}
\begin{paracol}{2}
\parblock{conda create -n pcot python=3.8 poetry}{Create a minimal conda environment containing just a known version of Python and the Poetry
dependency management system.}
\parblock{conda activate pcot}{Activate the environment.}
\parblock{poetry install}{Use Poetry to install all required packages as listed in the \texttt{pyproject.toml} and \texttt{poetry.lock}
files.}
\end{paracol}
\vspace{1em}
\noindent Now you should be able to run PCOT using the \texttt{pcot} command.


\subsection{Running PCOT inside PyCharm for development}
If you wish to run PCOT from inside IntelliJ's PyCharm IDE, follow these instructions after building an Anaconda environment as shown
above. First, tell PyCharm about the environment:
\begin{itemize}
\item Open PyCharm and open the PCOT directory as an existing project.
\item Open \textbf{Settings..} (Ctrl+Alt+S)
\item Select \textbf{Project:PCOT / Python Interpreter}
\item Select the cogwheel to the right of the Python Interpreter dropdown and then select \textbf{Add}.
\item Select \textbf{Conda Environment}.
\item Select \textbf{Existing Environment}.
\item Select the environment: it should be something like \texttt{anaconda3/envs/pcot/bin/python}.
\item Select \textbf{OK}.
\end{itemize}
Now set up the run configuration:
\begin{itemize}
\item Select \textbf{Edit Configurations} from the configurations drop down in the menu bar
\item Add a new configuration (the + symbol)
\item Set \textbf{Script Path} to \texttt{PCOT/src/pcot/main.py}
\item Make sure the interpreter is something like \texttt{Project Default (Python 3.8 (pcot))},
i.e. the Python interpreter of the pcot environment.
\end{itemize}
You should now be able to run and debug PCOT from inside PyCharm.

\subsection{Building a single file executable}
These brief instructions are for core PCOT developers only. They deal with creating single file executables for redistribution to users.

The following commands from inside an Anaconda shell should create a single file executable, assuming you have successfully created
a PCOT Anaconda install as given in Sec.~\ref{inst:conda} above.
\vspace{1em}
\begin{paracol}{2}
\parblock{conda activate pcot}{Ensure the PCOT environment is active.}
\parblock{pip3 install pyinstaller}{Install PyInstaller into the environment (we don't do this by default because not every user requires
it).}
\parblock{cd pyInstaller}{Go into the PyInstaller directory inside the main PCOT directory --- this contains the necessary specification
files.}
\parblock{pyinstaller file.spec}{Run PyInstaller: this will build a single, very large file called \texttt{dist/pcot}. It will
take a very long time.}
\end{paracol}
This process is prone to fail: small changes in PCOT's structure can make it difficult for PyInstaller to work. One particular problem
is ``hidden imports'' -- packages imported through non-standard means. For example, XForms cannot be
automatically detected when inside a single executable, so under a PyInstaller build they must
be explicitly listed in a file called \texttt{xformlist.txt} inside the main PCOT
package. Luckily, the PyInstaller specification file is just a Python program, and can automatically create this file. Read \texttt{file.spec} for the gory details.
Also, any non-standard widgets used in UI files need to be added to hidden imports, and any data files need to be added to a list of
such files.

