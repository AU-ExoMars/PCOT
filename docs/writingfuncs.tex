% Created by Jim Finnis
% Date Mon Jun 14 12:06:09 2021


\section{Writing custom \emph{expr} functions}
\label{writingfuncs}
This section covers writing custom functions for use in the expression evaluation node \emph{expr}.
All these functions have two arguments and return a single value. The arguments are:
\begin{itemize}
\item \textbf{args} is an array of mandatory positional arguments
\item \textbf{optargs} is an array of optional positional arguments which may follow the mandatory arguments.
\end{itemize}
All arguments and the return value are \texttt{Datum} objects, and we obtain the contained data by calling
the \texttt{get()} method with the appropriate \texttt{Type} object (type objects are static members of
\texttt{Datum}).
To register a function we use the \texttt{registerFunc()} method of \texttt{ExpressionEvaluator}, which
has the following arguments:
\begin{itemize}
\item function name
\item function description
\item array of \texttt{Parameter} objects describing the mandatory arguments
\item array of \texttt{Parameter} objects describing the optional arguments (which must be numeric)
\item the function itself
\end{itemize}
For built-in functions this is done in \texttt{ExpressionEvaluator}'s
constructor. For user functions, you must add an expression function hook from
a plugin file and register the functions in the hook (which is passed an
\texttt{ExpressionEvaluator}) --- see Sec.~\ref{customisation} for how to do
this.

The \texttt{Parameter} constructor takes the following arguments:
\begin{itemize}
\item parameter name
\item parameter description
\item tuple of permitted types for this parameter
\item default value (which must be numeric!) for optional parameters
\end{itemize}

\subsection{Example of a simple function}
Here is a simple function which generates a greyscale from an incoming image. There is an optional
numeric argument which is used as a flag to control how the function processes RGB images:
if the flag is true, 3-channel images are converted using OpenCV's
method\footnote{$0.299r + 0.587g + 0.114b$},
otherwise the mean of all channels is used. This latter method is always used when the image does not have
3 channels.

\begin{lstlisting}
def funcGrey(args, optargs):

    # Note that documentation strings are important: they are used in the
    # help system!
    
    """Greyscale conversion. If the optional second argument
    is nonzero, and the image has 3 channels, we'll use CV's
    conversion equation rather than just the mean."""

    # get the first argument as an ImageCube
    
    img = args[0].get(Datum.IMG)
    
    # get the image's sources, combined into a single set
    
    sources = set.union(*img.sources)

    # get the optional argument (or the default if not
    # provided).
    
    if optargs[0].get(Datum.NUMBER) != 0:

        # We are using the OpenCV method for 3 channels
        
        if img.channels != 3:
            # but there aren't 3 channels, raise an exception!
            raise XFormException('DATA', "Image must be RGB for OpenCV greyscale conversion")

        # generate a new image cube from the greyscale data, but keep the same
        # image mapping. Use the combined sources for this single channel.

        img = ImageCube(cv.cvtColor(img.img, cv.COLOR_RGB2GRAY), img.mapping, [sources])
    else:
        # create a transformation matrix specifying that the output is a single channel which
        # is the mean of all the channels in the source
        
        mat = np.array([1 / img.channels] * img.channels).reshape((1, img.channels))

        # use it to generate the image
        out = cv.transform(img.img, mat)

        # and turn this into an image cube
        img = ImageCube(out, img.mapping, [sources])
    
    # return the image cube as a Datum
    return Datum(Datum.IMG, img)
\end{lstlisting}

We can then register this function. This is a built-in function, so it is registered inside the constructor for
the \texttt{ExpressionEvaluator} owned by the \texttt{XFormExpr} type object:

\begin{lstlisting}
        self.registerFunc("grey", # name 
            "convert an image to greyscale", # description
            # a single mandatory parameter: an image
            [Parameter("image", "an image to process", Datum.IMG)],
            # a single optional parameter (which must be numeric):
            [Parameter("useCV",
               "if non-zero, use openCV greyscale conversion (RGB input only): 0.299*R + 0.587*G + 0.114*B",
                Datum.NUMBER, deflt=0)],
            funcGrey)
\end{lstlisting}

